\documentclass{jsarticle}
\usepackage{bm}
\newcommand{\argmax}{\mathop{\rm arg~max}\limits}
\begin{document}
\begin{itemize}
 \item トライ法
       \begin{itemize}
	\item 文字列探索アルゴリズム
	\item 木構造
	\item 一つのノードには文字種類数だけの大きさの配列
       \end{itemize}
 \item 形態素解析
       \begin{itemize}
	\item 形態素:語の最小単位
	\item 語の区切り、品詞、活用形などを求める処理
	\item 語の区切りの難しさ$\rightarrow$ビタビアルゴリズム
	\item 単語辞書にないもの$\rightarrow$未知語処理
	      \begin{itemize}

	       \item ビタビアルゴリズム
		     \begin{itemize}
		      \item 動的計画法
		      \item 形態素解析、品詞解析、固有表現認識などに
		      \item 文に対するラティス構造の最適パス選択に
		      \item 文頭ノードから、二つ以上のノードから繋がっているようなノー
	    ドまでのパスのコストとして、最小のものを選んでいく
		     \end{itemize}
	       \item 未知語処理
		     \begin{itemize}
		      \item 未知語を擬似ノードとして大きなコストを与える
		      \item 辞書中の単語に分割できなければ未知語
		      \item 松葉がに$\rightarrow$松葉かに
		     \end{itemize}
	      \end{itemize}
       \end{itemize}
 \item コーパス:文書を集めたもの
       \begin{itemize}
	\item 対訳コーパス
	\item 注釈付与コーパス(タグ(与える注釈)付きコーパス)
	      \begin{itemize}
	       \item 注釈付与コーパスの注釈を正解の基準とする
	       \item 素性の組み合わせの教師データとして利用
	      \end{itemize}
       \end{itemize}
 \item 言語モデル
       \begin{itemize}
	\item 文や表現の出現確率を与えるモデル
	\item n-gram言語モデル
	      \begin{itemize}
	       \item 単語の出現確率を与えるモデル
	       \item 記号の出現確率への影響を一定範囲の履歴に限定
	       \item n-1階マルコフモデル(直前のn-1個の単語から求める)
	       \item nの大きさに関して、表現の正確さとデータスパースネス
		     のトレードオフの関係
	      \end{itemize}
	\item 隠れマルコフモデル
	      \begin{itemize}
	       \item 観測される出力記号は、マルコフ過程における状態のう
		     ち、観測されない状態(隠れ状態)から確率的に出力されたものと過程
	       \item 英語の品詞タグ付け問題など
		     \begin{itemize}
		      \item 文を、品詞を隠れ状態としたときの各状態からの
			    出力の列とみなせる
		      \item 品詞列のマルコフ性の仮定のもと、
		      \begin{eqnarray*}\\
		       \bm{\hat{y}} & = & \argmax_{y} P(\bm{y}|\bm{x}) \\
		       &=& \argmax_{y}
			\frac{P(\bm{x}|\bm{y})P(\bm{y})}{P(\bm{x})} \\
		       &=& \argmax_{y}P(\bm{x}|\bm{y})P(\bm{y}) \\
		       &=& \argmax_{y} \prod_i{P(x_i|y_i)P(y_i|y_{i-1})}
		      \end{eqnarray*}
			    により、最も確率の高い品詞列(状態遷移)をビタ
			    ビアルゴリズムによって求める
		     \end{itemize}
	      \end{itemize}
	\item ナイーブベイズ
	      \begin{itemize}
	       \item 機械学習手法の一つ
	       \item \begin{eqnarray*}
		\hat{y} & = & \argmax_{y} P(y|\bm{x}) \\
		      &=& \argmax_{y}
		       \frac{P(\bm{x}|y)P(y)}{P(\bm{x})} \\
		      &=& \argmax_{y}P(\bm{x}|y)P(y) \\
&=& \argmax_{y} \prod_i{P(x_i|y)P(y)}
		     \end{eqnarray*}
	      \end{itemize}
	\item CRF
	      \begin{itemize}
	       \item 機械学習手法の一つ。対数線形モデル
	       \item 分類器の逐次適用における局所性の問題を解決
	       \item 入力文と品詞bigram$y_{i-1}$、$y_i$から素性を抽出す
		     る素性関数のみを元に$P(\bm{y}|\bm{x})$の最適化
	      \end{itemize}
	\item 固有表現認識
	      \begin{itemize}
	       \item 固有表現:地名、人名、組織名などの固有名に時間や数量などを加えたも
		     の
	       \item 情報抽出・除法検索において重要
	       \item BIOモデル
		     \begin{itemize}
		      \item 固有表現認識を系列ラベリングとして各単語にラベリング
		      \item 例えば、地名の始点(B-地名)、地名の続き
			    (I-地名)、いずれでもない(O)とラベリング
		      \item ビタビアルゴリズムで整合性を調べつつ、CRFなどによってラベリングを
			    行い、その組み合わせで固有表現を認識
		     \end{itemize}
	      \end{itemize}
       \end{itemize}
 \item 意味解析
       \begin{itemize}
	\item 意味、概念
	      \begin{itemize}
	       \item 語によって様々な対象を文節
	       \item 内包的定義
		     \begin{itemize}
		      \item 本質的な特徴・性質によりある概念を定義
		      \item A = $\{x|xは10以下の奇数\}$
		     \end{itemize}
	       \item 外延的定義
		     \begin{itemize}
		      \item その概念に含まれる具体例を列挙することによって定義
		      \item A = $\{1,3,5,7,9\}$
		     \end{itemize}
	       \item 上位概念、類/下位概念、種
		     \begin{itemize}
		      \item 内法的定義:特徴を受け継ぐ最も近い類(最近類)と最近類の他の種との差(種差)
		      \item (例)植物:光合成を行う生物
		      \item 外延的定義:種子植物、シダ植物、コケ植物
		     \end{itemize}
	      \end{itemize}
	\item 比喩
	      \begin{itemize}
	       \item 新たなことや抽象的なことを表現する際に、既存の具体的な物事を用いる
		     こと
	       \item 既存の具体物の特徴・属性(顕現性)に注目
	       \item 意味の拡張であることから、類犠牲の拡張のプロセスを利用することで比
		     喩表現の理解を導くことが可能
	       \item メタファー
		     \begin{itemize}
		      \item 比喩を明示しない比喩表現
		      \item (例)彼女はダイヤモンドだ
		     \end{itemize}
	       \item メトニミー
		     \begin{itemize}
		      \item 容器-中身、付属物-主体、作者-作品など
			    の近接性の関係を用いた比喩
		      \item (例)鍋を食べる
		      \item (例)白バイに捕まる
		     \end{itemize}
	      \end{itemize}
	\item シソーラス
	      \begin{itemize}
	       \item 意味の上位下位関係、同義関係を中心に語を体系的にまとめた辞書
	       \item 自然文によって意味を定義する辞書に比べコンピュータ
		     処理に適している
	       \item 自動抽出について
		     \begin{itemize}
		      \item Wikipediaなどウェブ上の進化する大規模辞書か
			    ら
		      \item 分布類似度計算によって同義関係を自動抽出			   
		     \end{itemize}
	      \end{itemize}
	\item 分布類似度
	      \begin{itemize}
	       \item 「文脈の似ている語は類似している」$\rightarrow$共起
		     する語が似ていれば類似している」という考えに基づく
		     尺度
	       \item ある語との自己相互情報量:
		     $PMI(x,y) = log{\frac{P(x,y)}{P(x)P(y)}}$
		     の高い語を関連語とし、ある2つの語が類似度の高い関連語
		     集合を持つ場合、その2つの語を類義語とする
	       \item 反義語と類義語の区別が困難
	      \end{itemize}
	\item 語義曖昧性解消(WSD)
	      \begin{itemize}
	       \item 文脈中の多義語の語義の選択する問題
	       \item 語の語義と文中の一定数の語のリストを付与した注釈付
		     与コーパスから教師あり学習として学習
	       \item (例)check、financeが文脈中にあればbank(銀行)と学習
	       \item Wikipediaの曖昧性回避ページなどを利用
	      \end{itemize}
       \end{itemize}
\end{itemize}
\end{document}