\documentclass[a4j,11pt]{jsarticle}
\usepackage{bm}
\newcommand{\argmax}{\mathop{\rm arg~max}\limits}
\begin{document}
\setcounter{section}{1}
\section{文字列・テキスト処理の基礎}
\begin{itemize}
 \item トライ法
       \begin{itemize}
	\item 文字列探索アルゴリズム
	\item 木構造
	\item 一つのノードには文字種類数だけの大きさの配列
	\item 各配列に「rootからたどってきた文字列$+$配列の文字」の文字列へのポインタ
	\item 遷移先がキー(辞書に登録されている単語)ならマークをつける
	\item 例: root[く] $\rightarrow$ く[る] $\rightarrow$ くる*[ま] $\rightarrow$ くるま*
	\item 日本語は文字種類数が数千種あるので数段で記憶領域が爆発。
	      配列を文字ではなくバイトやビットで作る
       \end{itemize}
\end{itemize}

\section{系列の解析(1)}
\begin{itemize}
 \item 形態素解析
       \begin{itemize}
	\item 形態素:語の最小単位(日本語なら単語。接頭辞・接尾辞も便宜的に語と見る)
	\item 語の区切り、品詞、活用形などを求める処理
	\item 語の区切りの難しさ$\rightarrow$ビタビアルゴリズム
	\item 単語辞書にないもの$\rightarrow$未知語処理
	      \begin{itemize}

	       \item ビタビアルゴリズム
		     \begin{itemize}
		      \item 動的計画法
		      \item 形態素解析、品詞解析、固有表現認識などに
		      \item 文に対するラティス構造の最適パス選択に
		      \item 文頭に(文字数で)近いノードから順に、文頭ノードから各ノードへのコストが
			    最小となるようなパスを選んでいく
		     \end{itemize}
	       \item 未知語処理
		     \begin{itemize}
		      \item 未知語を擬似ノードとして大きなコストを与える
			    (漢字連続、カタカナ連続などを1語とする)
		      \item 辞書中の単語に分割できなければ未知語
		      \item 既知の単語へ帰着(松葉がに$\rightarrow$松葉かに)
		      \item 単語辞書の自動拡充(Wikipediaの見出しなどを利用)
		      \item 「フットサル」は「サル」の一種?(類似度計算)
		     \end{itemize}
	      \end{itemize}
       \end{itemize}
\end{itemize}

\section{コーパスに基づく自然言語処理}
\begin{itemize}
 \item コーパス:文書を集めたもの
       \begin{itemize}
	\item 生コーパス
	\item 均衡コーパス: 新聞、書籍、雑誌など様々なジャンルのテキストをバランスよく収集したもの
	\item 対訳コーパス(パラレルコーパス)
	\item コンパラブルコーパス: 対訳関係ではないが、同じトピックに関する2言語の文書対を収集したもの
	\item 注釈付与コーパス(タグ付きコーパス)
	      \begin{itemize}
	       \item 注釈(タグ): 形態素, 構文, 固有表現, 語の意味, 省略照応, 談話関係, テキスト分類など
	       \item 注釈付与コーパスの注釈を正解の基準とする
	       \item 素性の組み合わせの教師データとして利用
	      \end{itemize}
       \end{itemize}
 \item 言語モデル
       \begin{itemize}
	\item 文や表現の出現確率を与えるモデル
	\item n-gram言語モデル
	      \begin{itemize}
	       \item 単語の出現確率を与えるモデル
	       \item 記号の出現確率への影響を一定範囲の履歴に限定
	       \item n-1階マルコフモデル(直前のn-1個の単語を参照)
	       \item 例:bigram
		     \begin{eqnarray*}
		      P(私は本を買った) = P(私|文頭) \times P(は|私) \times P(本|は) \\
		       \times P(を|本) \times P(買った|を)
		     \end{eqnarray*}
	       \item nを大きくすれば正確なモデルが作れるが、
		     データスパースネス(確率0の要素が多くなる)の問題が深刻に\\
		     最近は5-gramあたりが有効に
	      \end{itemize}
       \end{itemize}
 \item ナイーブベイズ
       \begin{itemize}
	\item 機械学習手法の一つ。分類(教師あり学習)手法
	\item 素性ベクトル$\bm{x}$、クラスラベル$y$、各素性$x_i$は他の素性$x_j$と独立と仮定
	      \begin{eqnarray*}
	       \hat{y} &=& \argmax_{y} P(y|\bm{x}) \\
	       &=& \argmax_{y}
		\frac{P(\bm{x}|y)P(y)}{P(\bm{x})} \\
	       &=& \argmax_{y}P(\bm{x}|y)P(y) \\
	       &=& \argmax_{y} \prod_i{P(x_i|y)P(y)}
	      \end{eqnarray*}
       \end{itemize}
\end{itemize}

\section{系列の解析(2)}
\begin{itemize}
 \item 隠れマルコフモデル(HMM)
       \begin{itemize}
	\item 観測される出力記号は、マルコフ過程における状態のうち、
	      観測されない状態(隠れ状態)から確率的に出力されたものと仮定
	\item 英語の品詞タグ付け問題など
	      \begin{itemize}
	       \item 文を、品詞を隠れ状態としたときの各状態からの出力の列とみなせる
	       \item 品詞列のマルコフ性
		     \[
		      P(\bm{y}) = \prod_i P(y_i|y_{i-1})
		     \]
		     と、単語出力がその品詞だけに依存する
		     \[
		      P(\bm{x}|\bm{y}) = \prod_i P(x_i|y_i)
		     \]
		     ことを仮定すると、
		     \begin{eqnarray*}
		      \bm{\hat{y}} &=& \argmax_{y} P(\bm{y}|\bm{x}) \\
		      &=& \argmax_{y} \frac{P(\bm{x}|\bm{y})P(\bm{y})}{P(\bm{x})} \\
		      &=& \argmax_{y} P(\bm{x}|\bm{y})P(\bm{y}) \\
		      &=& \argmax_{y} \prod_i{P(x_i|y_i)P(y_i|y_{i-1})}
		     \end{eqnarray*}
		     により、入力文字列$\bm{x}$に対する最も確率の高い
		     品詞列(状態遷移)$\bm{y}$をビタビアルゴリズムによって求められる
	      \end{itemize}
       \end{itemize}
 \item CRF(条件付き確率場)
       \begin{itemize}
	       \item 機械学習手法の一つ。対数線形モデル
	       \item 分類器の逐次適用における局所性の問題を解決
	       \item 入力文と品詞bigram$y_{i-1}$、$y_i$から素性を抽出す
		     る素性関数のみを元に$P(\bm{y}|\bm{x})$の最適化
       \end{itemize}
 \item 固有表現認識
       \begin{itemize}
	\item 固有表現:地名、人名、組織名などの固有名および、時間(2015 年)や数量(3 個)など
	\item どこまでが一続きの固有表現か、どの固有表現か
	      (「ローマ」は地名?サカーチーム名(組織名)?あるいは人名?)
	\item 情報抽出・情報検索において重要
	\item BIOモデル
	      \begin{itemize}
	       \item 固有表現認識を系列ラベリングとして各単語にラベリング
	       \item 例えば、地名の始点(B-地名)、地名の続き
		     (I-地名)、いずれでもない(O)とラベリング
	       \item ビタビアルゴリズムで整合性(固有表現はIから始まらない)を調べつつ、
		     CRFなどによってラベリングを行い、その組み合わせで固有表現を認識
	      \end{itemize}
       \end{itemize}
\end{itemize}

\section{意味の解析(1)}
\begin{itemize}
 \item 意味、概念
       \begin{itemize}
	\item 語によって様々な対象を分節
	      \begin{itemize}
	       \item 対象に語を与えることで、その対象に対応する概念を作り出す(切り出す)
	       \item 例:「わびさび」という概念は英語圏にない
	      \end{itemize}
	\item 内包的定義
	      \begin{itemize}
	       \item 本質的な特徴・性質によりある概念を定義
	       \item A = $\{x|xは10以下の奇数\}$
	       \item 芸術: 美を創造・表現しようとする人間活動, およびその作品(三省堂/大辞林)
	      \end{itemize}
	\item 外延的定義
	      \begin{itemize}
	       \item その概念に含まれる具体例を列挙することによって定義
	       \item A = $\{1,3,5,7,9\}$
	       \item 芸術: 建築, 彫刻, 音楽, 文学, 演劇, 舞踊, 映画など
	      \end{itemize}
	\item 上位概念、類/下位概念、種
	      \begin{itemize}
	       \item 上位概念(類): その概念を含む、より一般的・総称的・抽象的な概念
	       \item 下位概念(種): その概念に関する、より特定・個別・具体的な概念
	       \item 内法的定義: 特徴を受け継ぐ最も近い類(最近類)と最近類の他の種との差(種差)
	       \item 外延的定義: 下位概念の列挙
	       \item (例)植物: 光合成を行う生物(上位概念:生物)。
		     種子植物、シダ植物、コケ植物など(下位概念)。
	      \end{itemize}
       \end{itemize}
 \item 比喩
       \begin{itemize}
	\item 新たなことや抽象的なことを表現する際に、既存の具体的な物事を用いる
	      こと
	\item 既存の具体物の特徴・属性(顕現性)に注目
	\item 意味の拡張であることから、類犠牲の拡張のプロセスを利用することで比
	      喩表現の理解を導くことが可能
	\item 直喩
	      \begin{itemize}
	       \item 比喩を明示する表現(〜のようだ)を伴うもの
	       \item (例) 彼女はダイヤモンドのようだ
		     (ダイヤモンドの「輝く」という特徴・属性(顕現性)が取り出されている)
	      \end{itemize}
	\item メタファー(隠喩)
	      \begin{itemize}
	       \item 比喩を明示しない比喩表現
	       \item (例)彼女はダイヤモンドだ
	      \end{itemize}
	\item メトニミー(換喩)
	      \begin{itemize}
	       \item 容器-中身、付属物-主体、作者-作品など
		     の近接性の関係を用いた比喩
	       \item (例)鍋を食べる
	       \item (例)白バイに捕まる
	      \end{itemize}
       \end{itemize}
 \item シソーラス
       \begin{itemize}
	\item 意味の上位下位関係、同義関係を中心に語を体系的にまとめた辞書
	\item 自然文によって意味を定義する辞書に比べコンピュータ
	      処理に適している
	\item 自動抽出について
	      \begin{itemize}
	       \item Wikipediaなどウェブ上の進化する大規模辞書から
	       \item 分布類似度計算によって同義関係を自動抽出
	      \end{itemize}
       \end{itemize}
 \item 分布類似度
       \begin{itemize}
	\item 「文脈の似ている語は類似している」$\rightarrow$
	      「共起する語が似ていれば類似している」
	      という考えに基づく尺度\\
	      (共起: 2つの語がある範囲[同一文書内,同一文内,前後10語以内,係り受け関係など]
	      で共に[同時に]出現すること)
	\item 類義語の自動獲得に利用
	\item ある語との自己相互情報量:
	      $\mathrm{PMI}(x,y) = \log{\frac{P(x,y)}{P(x)P(y)}}$
	      の高い語を関連語とし、ある2つの語が類似度の高い関連語
	      集合を持つ場合、その2つの語を類義語とする
	\item 反義語と類義語の区別が困難
       \end{itemize}
 \item 語義曖昧性解消(WSD)
       \begin{itemize}
	\item 文脈中の多義語の語義を選択する問題
	\item 語の語義と文中の一定数の語のリストを付与した注釈付
	      与コーパスから教師あり学習として学習
	\item (例)check、financeが文脈中にあればbank(銀行)と学習
	\item Wikipediaの曖昧性回避ページなどを利用
       \end{itemize}
\end{itemize}

\section{構文の解析(1)}
\begin{itemize}
 \item 
\end{itemize}

\end{document}
